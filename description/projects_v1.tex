\resheading{Projects}
  \begin{itemize}[leftmargin=*]

    \item \ressubsingleline{Formal Collision Avoidance Analysis for Rigorous Building of Autonomous Marine Vehicles}{}{}
    {\small
     \begin{itemize}
         \item{Description: Collision avoidance is an essential ingredient to ensure safety for autonomous marine vehicles (AMVs). We investigate the fundamental rules of collision avoidance for AMVs. In the investigation, we present a formal model using timed automata to analyze the completeness and consistency of the rules, and then enumerate all the encountering scenarios and present the analysis from various perspectives.}
     \end{itemize}
    }

    \item \ressubsingleline{LiDAR Based Navigable Region Detection for Unmanned Surface Vehicles}{}{}
    {\small
     \begin{itemize}
         \item{Description: Detection of the navigable regions for the unmanned surface vehicles (USVs) sailing on the narrow rivers is very important. We propose a scheme to process 3D LiDAR point clouds to achieve an accurate and robust navigable regions detection. This scheme consists of three parts: 1)Classification of river objects, e.g., river banks, bridges and vegetation, by a deep learning method; 2)Extraction of water surface by an improved convex polygon fitting method; 3)Filtering out the noise point clouds on the water surface by a customized particle filter to get the fine navigable region.}
     \end{itemize}
    }

    \item \ressubsingleline{Remote Motive Maritime Objects Detection Based on Multi-sensor Fusion Method}{}{}
    {\small
     \begin{itemize}
         \item{Description: The marine environment is complex, so it is difficult for a single sensor to provide comprehensive environmental information for the unmanned surface vehicles. In this project, we propose a multi-sensor fusion method, which can fuse the data of binocular camera, LiDAR, IMU and GPS, to measure the distance and moving state of the maritime objects.}
     \end{itemize}
    }

    \item \ressubsingleline{SLAM in complex river scene based on vision sensor and LiDAR}{}{}
    {\small
     \begin{itemize}
         \item{Description: Both mapping and localization are the crucial steps of autonomous navigation for USVs. GPS may not always be available, because it is vulnerable to natural interference. In some complex river scenes, vegetation, bridges and other obstacles will affect the localization accuracy of GPS, so it is not reliable for unmanned vessel to use GPS for positioning. In order to improve localization and mapping accuracy of unmanned vessel in complex river scenes, we try to improve SLAM algorithm by fusing the data of LiDAR and visual sensor.}
     \end{itemize}
    }

  \end{itemize}
