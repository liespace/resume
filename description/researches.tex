\resheading{研究经历}
  \begin{itemize}[leftmargin=*]
    \item \ressubsingleline{针对自动驾驶的运动规划}{}{2018.09 -- 目前}{\small
    	\begin{itemize}
    		\item{硕士阶段的研究领域是自动驾驶的运动规划,研究方向是深度学习方法与经典运动规划方法的结合:利用深度卷积网络(如YOLO、R-CNN等)为基于采样的路径规划算法(如RRT*等)提供高效的启发。}
    		\item{研究用仿真实验平台基于CARLA和ROS框架搭建,神经网络则基于Tensorflow和Keras框架搭建。仿真平台展示:\url{https://drive.google.com/open?id=1BUK_68Hakp6KNbl9BVbhMp8W29Q4aNgo}。}
		    \item{2018.09 -- 2019.05,基于ResNet50网络对参考路径(道路中线)进行适应性修正,进一步提高了RRT*路径算法在结构化路面(如高速公路、城市道路等)上的规划效率和质量。相关成果已经申请一项发明专利。}
	        \item{2019.06 -- 目前,将研究方法应用到非结构化路面(停车场,乡村道路等)上的自动泊车的问题。基于VGG19网络直接生成参考路径,实现了单一RRT*算法实时、高质量的路径规划效果。阶段成果已撰写成论文投稿至ICAPS(规划领域顶会):\textbf{Fan Li}, Yunxiao Shan, Mingyue Cui, Kai huang: \textbf{\textit{DeepPlanning: Deep Learning-based Planning Method for Autonomous Parking}} (Under review at ICAPS)。}
         \end{itemize}}

  \end{itemize}
